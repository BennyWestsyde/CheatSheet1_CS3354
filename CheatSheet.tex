\documentclass[10pt]{article}
\usepackage[utf8]{inputenc}
\usepackage[letterpaper, margin=0.1in, landscape]{geometry}
\usepackage{tcolorbox}
\usepackage{multicol}
\usepackage{enumitem}
% set the font to a monospace font

%\tcbuselibrary{breakable}
\setitemize{noitemsep,topsep=0pt,parsep=0pt,partopsep=0pt,leftmargin=*}
\setlength{\columnsep}{1pt} % Adjust column separation

% Define a custom box command
\newtcolorbox{block}[1][]{
  colback=white, % Background color
  coltitle=white, % Title color
  colbacktitle=black, % Title background color
  colframe=black, % Border color
  fonttitle=\bfseries\tiny, % Title font
  fontupper=\tiny, % Text font
  boxrule=0.5mm, % Border width
  sharp corners, % No rounded corners
  boxsep=1pt, % Padding
  left=1pt, % Padding
  right=1pt, % Padding
  top=1pt, % Padding
  bottom=1pt, % Padding 
  title=#1, % Set the title to the first argument
  parbox=false, % Allow text to break across pages
  beforeafter skip=0pt
}
\begin{document}
\ttfamily
% Set up multicol for tightly packed boxes
\begin{multicols}{5}
  
\begin{block}[Estimation and Planning Mistakes]
  \begin{itemize}
    \item \textbf{Underestimating complexity, cost, and/or schedule}
    \begin{itemize}
      \item Use historical data and expert judgment to estimate accurately.
    \end{itemize}
    \item \textbf{Abandoning planning under pressure}
    \begin{itemize}
      \item Stick to planning to avoid chaotic code-and-fix mode.
    \end{itemize}
    \item \textbf{Overly aggressive schedules}
    \begin{itemize}
      \item Set realistic schedules based on historical data and project complexity.
    \end{itemize}
    \item \textbf{Wasting time in the “fuzzy front end”}
    \begin{itemize}
      \item Streamline the approval and budgeting process.
    \end{itemize}
  \end{itemize}
  \end{block}
  
  \begin{block}[Communication and Stakeholder Engagement Mistakes]
  \begin{itemize}
    \item \textbf{Poor communication}
    \begin{itemize}
      \item Hold regular meetings and ensure clear documentation.
    \end{itemize}
    \item \textbf{Not engaging stakeholders}
    \begin{itemize}
      \item Include stakeholders in planning and review sessions.
    \end{itemize}
    \item \textbf{Insufficient user input}
    \begin{itemize}
      \item Ensure active involvement of end-users throughout the project.
    \end{itemize}
  \end{itemize}
  \end{block}
  
  \begin{block}[Project Management Mistakes]
  \begin{itemize}
    \item \textbf{Lack of oversight/poor project management}
    \begin{itemize}
      \item Appoint experienced project managers and conduct regular reviews.
    \end{itemize}
    \item \textbf{Adding developers to a late project}
    \begin{itemize}
      \item Avoid adding developers late in the project to prevent further delays.
    \end{itemize}
  \end{itemize}
  \end{block}
  
  \begin{block}[Quality and Risk Management Mistakes]
  \begin{itemize}
    \item \textbf{Poor quality workmanship}
    \begin{itemize}
      \item Implement quality assurance processes and conduct regular code reviews.
    \end{itemize}
    \item \textbf{No risk management}
    \begin{itemize}
      \item Identify risks early and develop mitigation plans.
    \end{itemize}
    \item \textbf{Ignoring system performance requirements}
    \begin{itemize}
      \item Define and monitor performance requirements throughout the project.
    \end{itemize}
    \item \textbf{Poorly planned/managed transitions}
    \begin{itemize}
      \item Develop detailed transition plans and involve all relevant parties.
    \end{itemize}
  \end{itemize}
  \end{block}

  \begin{block}[Recursive vs Incremental vs Iterative Development]
  \begin{itemize}
    \item \textbf{Recursive}
    \begin{itemize}
      \item Repeatedly breaking down a problem into smaller parts until it is simple enough to solve.
      \item Example: Divide and conquer.
    \end{itemize}
    \item \textbf{Incremental}
    \begin{itemize}
      \item Start by building the core functionality and then add features in subsequent increments.
      \item Example: Agile.
    \end{itemize}
    \item \textbf{Iterative}
    \item \begin{itemize}
      \item Develop a system through repetition of cycles (iterations)
      \item Example: Scrum.
    \end{itemize}
  \end{itemize}
  \end{block}

  
  \begin{block}[Unified Process]
    \begin{itemize}
      \item \textbf{Workflows}
      \begin{itemize}
        \item Defines activities in process.
        \item Each activity has inputs and outputs.
      \end{itemize}
      \item \textbf{Phases}
      \begin{itemize}
        \item Inception
        \item Elaboration
        \item Construction
        \item Transition
      \end{itemize}
    \end{itemize}
  \end{block}
  
  \begin{block}[Cynefin Framework]
    \begin{itemize}
      \item \textbf{Simple}
      \begin{itemize}
        \item Cause and effect are obvious.
        \item Best practice.
      \end{itemize}
      \item \textbf{Complicated}
      \begin{itemize}
        \item Cause and effect are discoverable.
        \item Good practice.
      \end{itemize}
      \item \textbf{Complex}
      \begin{itemize}
        \item Cause and effect are only obvious in hindsight.
        \item Emergent practice.
      \end{itemize}
      \item \textbf{Chaotic}
      \begin{itemize}
        \item No cause and effect relationship.
        \item Novel practice.
      \end{itemize}
    \end{itemize}
  \end{block}

  \begin{block}[Predictive vs Adaptive Development]
    \begin{itemize}
      \item \textbf{Predictive}
      \begin{itemize}
        \item Plan-driven.
        \item Requirements are stable.
        \item Example: Waterfall.
      \end{itemize}
      \item \textbf{Adaptive}
      \begin{itemize}
        \item Change-driven.
        \item Requirements are volatile.
        \item Example: Agile.
      \end{itemize}
    \end{itemize}
  \end{block}
  
  \begin{block}[Waterfall Model]
    \begin{itemize}
      \item \textbf{Requirements}
      \begin{itemize}
        \item Define system requirements.
      \end{itemize}
      \item \textbf{Design}
      \begin{itemize}
        \item Develop system architecture.
      \end{itemize}
      \item \textbf{Implementation}
      \begin{itemize}
        \item Write and test code.
      \end{itemize}
      \item \textbf{Verification}
      \begin{itemize}
        \item Test system.
      \end{itemize}
      \item \textbf{Maintenance}
      \begin{itemize}
        \item Fix bugs and add features.
      \end{itemize}
    \end{itemize}
  \end{block}

\begin{block}[Agile Manifesto]
    \begin{itemize}
        \item Individuals and interactions over processes and tools.
        \item Working software over comprehensive documentation.
        \item Customer collaboration over contract negotiation.
        \item Responding to change over following a plan.
    \end{itemize}
\end{block}

\begin{block}[Agile Principles]
    \begin{itemize}
        \item Satisfy the customer with continuous delivery.
        \item Welcome changing requirements.
        \item Frequent delivery of working software.
        \item Daily collaboration between business and developers.
        \item Build projects around motivated individuals.
        \item Face-to-face conversation for communication.
        \item Working software as progress measure.
        \item Promote sustainable Dev.
        \item Continuous attention to technical excellence.
        \item Simplicity is essential.
        \item Best architectures emerge from self-organizing teams.
        \item Regular reflection and adjustment.
    \end{itemize}
\end{block}

\begin{block}[Scrum Roles]
    \begin{itemize}
        \item Product Owner:
        \begin{itemize}
            \item Maximizes product value.
            \item Develops and communicates Product Goal.
            \item Creates and prioritizes Product Backlog.
            \item Ensures transparency and understanding of Backlog.
            \item One person, not a committee, with leadership role.
        \end{itemize}
        \item Scrum Master:
        \begin{itemize}
            \item Facilitates Scrum process, resolves impediments.
            \item Creates self-organization environment.
            \item Captures empirical data, shields team from distractions.
            \item Enforces timeboxes, keeps artifacts visible.
            \item Promotes improved practices, has leadership role.
        \end{itemize}
        \item Dev Team:
        \begin{itemize}
            \item Develops product, self-organizing, cross-functional.
            \item No titles, no sub-teams, no specialized roles.
            \item Long-term, full-time membership, 7 ± 2 members.
        \end{itemize}
    \end{itemize}
\end{block}



\begin{block}[Scrum Cycle]
    \begin{itemize}
        \item \textbf{Sprint Planning}
        \begin{itemize}
            \item Product Owner presents Product Backlog.
            \item Dev Team selects items for Sprint Backlog.
            \item Sprint Goal is defined.
        \end{itemize}
        \item \textbf{Daily Scrum}
        \begin{itemize}
            \item 15-minute meeting.
            \item Dev Team plans work for next 24 hours.
            \item Scrum Master enforces timebox.
        \end{itemize}
        \item \textbf{Sprint Review}
        \begin{itemize}
            \item Product Owner presents completed work.
            \item Dev Team demonstrates work.
            \item Stakeholders provide feedback.
        \end{itemize}
        \item \textbf{Sprint Retrospective}
        \begin{itemize}
            \item Dev Team reflects on Sprint.
            \item Scrum Master facilitates discussion.
            \item Team identifies improvements.
        \end{itemize}
    \end{itemize}    
\end{block}

\begin{block}[Scrum Artifacts]
    \begin{itemize}
        \item \textbf{Product Backlog}
        \begin{itemize}
            \item Prioritized list of features.
            \item Updated regularly.
            \item Visible to all stakeholders.
            \item Owned by Product Owner.
        \end{itemize}
        \item \textbf{Sprint Backlog}
        \begin{itemize}
            \item List of tasks for current Sprint.
            \item Owned by Dev Team.
            \item Updated daily.
            \item Created during Sprint Planning Meeting.
            \item Decomposed from Product Backlog.
        \end{itemize}
        \item \textbf{Burndown Charts}
        \begin{itemize}
            \item Graphical representation of work remaining.
            \item Updated daily.
            \item Shows progress towards Sprint Goal.
            \item Helps identify issues early.
            \item Used to forecast project completion.
        \end{itemize}
    \end{itemize}
\end{block}

\begin{block}[Increment \& Done]
    \begin{itemize}
        \item A formal description of the state of the Increment when it meets the quality measures required for the product.
        \item The moment a Product Backlog item meets the Definition of Done, an Increment is born.
        \item If a Product Backlog item does not meet the Definition of Done, it cannot be released or even presented at the Sprint Review.
    \end{itemize}
\end{block}

\begin{block}[No Silver Bullet]
    \begin{itemize}
        \item Scrum will not solve your problems.
        \item Scrum will make your problems visible.
        \item You will have to solve your problems.
    \end{itemize}
\end{block}


\begin{block}[Accidental vs Essential Complexity]
\begin{itemize}
  \item Essential complexity:
    - Inherently difficult problems with no known solution.
  \item Necessary accidental complexity:
    - Example: project management.
  \item Unnecessary accidental complexity:
    - Waste, Lean, MEI (minimum essential information).
\end{itemize}
\end{block}

\begin{block}[Best/Good/Recommended Practices]
\begin{itemize}
  \item "Best Practice":
    - Consistently improves productivity, cost, schedule, quality, user satisfaction, predictability.
  \item Best Practices (Glass, 2004):
    - Dev teams repeat mistakes.
    - Best practice documents regurgitate textbook material.
    - Growing field's wisdom not increasing.
\end{itemize}
\end{block}


\begin{block}[Agile Sweet Spots]
\begin{itemize}
  \item Dedicated developers.
  \item Experienced developers.
  \item Small co-located team.
  \item Tools for testing and configuration management.
  \item Easy user access.
  \item Short increments and frequent delivery.
\end{itemize}
\end{block}

\begin{block}[Requirements Volatility]
\begin{itemize}
  \item \textbf{Failure to consider how requirements will change}
  \begin{itemize}
    \item Requirements change about 2\% per month for typical project.
    \item Change rates of 35-50\% for large projects.
    \item Typical software project experiences 25\% change in requirements.
  \end{itemize}
\end{itemize}
\end{block}


\begin{block}[Requirements Elicitation]
\begin{itemize}
  \item \textbf{Interviews}
  \begin{itemize}
    \item \underline{Structured interviews}
    \begin{itemize}
      \item Specific preplanned questions are asked.
    \end{itemize}
    \item \underline{Unstructured interview}
    \begin{itemize}
      \item Questions are posed in response to the answers received.
    \end{itemize}
  \end{itemize}
  \item \textbf{Questions}
  \begin{itemize}
    \item \underline{Open-ended questions}
    \begin{itemize}
      \item Questions are posed to encourage the client to provide more information.
    \end{itemize}
    \item \underline{Closed-ended questions}
    \begin{itemize}
      \item Questions are posed to answer specific questions.
    \end{itemize}
  \end{itemize}
\end{itemize}
\end{block}

\begin{block}[Requirements Design]
  \begin{itemize}
    \item \textbf{Functional Requirements}
    \begin{itemize}
      \item Define system behavior.
      \item Define what system should do.
    \end{itemize}
    \item \textbf{Non-Functional Requirements}
    \begin{itemize}
      \item Describe system properties and constraints.
      \item Define how system should do it.
    \end{itemize}
  \end{itemize}
\end{block}

\begin{block}[Requirements Analysis]
  \begin{itemize}
    \item \textbf{Functional Decomposition}
    \begin{itemize}
      \item Breaks down system into smaller components.
      \item Each component has a specific function.
    \end{itemize}
    \item \textbf{Data Flow Diagrams}
    \begin{itemize}
      \item Shows how data flows through system.
      \item Identifies sources and destinations of data.
    \end{itemize}
    \item \textbf{State Diagrams}
    \begin{itemize}
      \item Shows how system responds to events.
      \item Identifies states system can be in.
    \end{itemize}
    \item \textbf{Entity-Relationship Diagrams}
    \begin{itemize}
      \item Shows how data is related in system.
      \item Identifies entities and relationships.
    \end{itemize}
  \end{itemize}
\end{block}

\begin{block}[Characteristics of a good requirement]
\begin{itemize}
  \item \textbf{Necessary}
  \begin{itemize}
    \item Defines an essential capability, characteristic, constraint, and/or quality factor.
    \item If removed, a deficiency will exist.
  \end{itemize}
  \item \textbf{Implementation Free}
  \begin{itemize}
    \item Avoids placing unnecessary constraints on the architectural design.
    \item States what is required, not how the requirement will be satisfied.
  \end{itemize}
  \item \textbf{Unambiguous}
  \begin{itemize}
    \item The requirement is stated in such a way so that it can be interpreted in only one way.
    \item The requirement is stated simply and is easy to understand.
    \end{itemize}
  \item \textbf{Consistent}
  \begin{itemize}
    \item The requirement is free of conflicts with other requirements.
  \end{itemize}
  \item \textbf{Complete}
  \begin{itemize}
    \item The stated requirement needs no further amplification because it is measurable and sufficiently describes the capability and characteristics to meet the stakeholder's need.
  \end{itemize}
  \item \textbf{Singular}
  \begin{itemize}
    \item The requirement statement includes only one requirement with no use of conjunctions.
  \end{itemize}
  \item \textbf{Feasible}
  \begin{itemize}
    \item The requirement is technically achievable, does not require major technology advances, and fits within system constraints (e.g., cost, schedule, technical, legal, regulatory) with acceptable risk.
  \end{itemize}
  \item \textbf{Traceable}
  \begin{itemize}
    \item The requirement is upwards traceable to specific documented stakeholder statement(s) of need, higher tier requirement, or other source (e.g., a trade or design study).
    \item The requirement is also downwards traceable to the specific requirements in the lower tier requirements specification or other system definition artefacts.
  \end{itemize}
  \item \textbf{Verifiable}
  \begin{itemize}
    \item The requirement has the means to prove that the system satisfies the specified requirement.
  \end{itemize}
\end{itemize}
\end{block}

\begin{block}[Characteristics of a good \textbf{set of} requirements]
\begin{itemize}
  \item \textbf{Complete}
  \begin{itemize}
    \item Needs no further amplification.
    \item Acceptable timeframe for TBD items.
  \end{itemize}
  \item \textbf{Consistent}
  \begin{itemize}
    \item No contradictory requirements.
    \item No duplicated requirements.
    \item Same term used for same item.
  \end{itemize}
  \item \textbf{Affordable}
  \begin{itemize}
    \item Can be satisfied by a feasible solution.
    \item Within life cycle constraints.
  \end{itemize}
  \item \textbf{Bounded}
  \begin{itemize}
    \item Maintains identified scope.
    \item Does not increase beyond what is needed.
  \end{itemize}
\end{itemize}
\end{block}

\begin{block}[Use Cases vs User Stories]
  \begin{itemize}
    \item \textbf{Use cases}
    \begin{itemize}
      \item A set of scenarios that identify a thread of usage for the system to be constructed.
      \item Tells a stylized story about how an end user interacts with the system under a specific set of circumstances.
      \item Captures a contract that describes the system's behavior under various conditions as the system responds to a request from one of its stakeholders.
      \item NOT OBJECT ORIENTED.
    \end{itemize}
    \item \textbf{User stories}
    \begin{itemize}
      \item A promise to have a discussion; not every detail needs to be included.
      \item Describes functionality that will be valuable to either a user or purchaser of a system.
      \item \underline{Card}
      \begin{itemize}
        \item Written description of the story used for planning and as a reminder.
      \end{itemize}
      \item \underline{Conversation}
      \begin{itemize}
        \item About the story that serve to flesh out the details of the story.
      \end{itemize}
      \item \underline{Confirmation}
      \begin{itemize}
        \item Details that can be used to determine when a story is complete.
      \end{itemize}
    \end{itemize}
  \end{itemize}
\end{block}

\begin{block}[Use Case Format]
  \begin{itemize}
    \item Our user starts by [action]
    \item Then, the system [response]
    \item The user then [reacts]
    \item Finally, the system [result]
    \item Leaving the user [result]
  \end{itemize}
\end{block}


\begin{block}[User Story Format]
  \begin{itemize}
    \item \textbf{As a} [role]
    \item \textbf{I want} [feature]
    \item \textbf{So that} [benefit]
  \end{itemize}
\end{block}


\begin{block}[Facts and Figures]
  \begin{itemize}
    \item \textbf{Requirements volatility}
    \begin{itemize}
      \item 2\% per month for typical project.
      \item 35-50\% for large projects.
      \item 25\% change in requirements for typical software project.
    \end{itemize}
    \item \textbf{Scrum Team Size}
    \begin{itemize}
      \item 7 ± 2 members or < 10 members.
    \end{itemize}
  \end{itemize}
  \end{block}


\end{multicols}
\end{document}